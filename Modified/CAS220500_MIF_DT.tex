\subsubsection{Cost Category 22.5: Fuel Handling and Storage}

Consists of: 22.5.1 chamber exhaust gas handling and processing equipment (H\&P); 22.5.2 purge and cover gas H\&P; 22.5.3 primary coolant stream H\&P; 22.5.4 purification and isotope separation; 22.5.5 tritium, deuterium and DT storage; 22.5.6 atmospheric tritium recovery.\\

Various previous systems and cost analyses are available as extrapolation points including ARIES, Starfire and ITER. For this report, ITER is used as the benchmark, then linearly scaled by $P_E$, scaled for inflation, and a learning curve credit of 0.8 is applied, equating to a factor of 0.5 for a tenth-of-a-kind system. The resulting total cost is 88.7 M USD. A breakdown of costs for each susbsystem for ITER and this concept can be found in table \ref{tab:fuel}.\\

The fuel handling and storage system in fusion reactors is responsible for the online processing of fuel isotopes, encompassing extraction, recovery, purification, preparation, and storage. Separate from fuel injection (Account 22.03.04), this system processes materials from various sources including liquid breeders, chamber gases, and tritium-bearing streams, ensuring safe tritium levels in power core and heat transfer fluids under normal and emergency conditions.\\

While most equipment is commercially available, heightened reliability and integrity standards, especially for tritium-containing materials, lead to higher costs. For example, atmospheric tritium recovery systems are costly due to the need for rapid and efficient tritium concentration reduction. This account's functionality and requirements are well-understood from existing and developing fusion facilities, with revisions from previous versions. Cost increases are expected due to higher reliability demands for power plant applications, but this may be offset by learning curve effects in subsequent plant constructions. The recommended Life-Cycle Analysis (LSA) factors for all accounts are 0.85 (LSA1) and 0.94 (LSA 2), reflecting these considerations.


\begin{table}
    \centering
    \begin{tabular}{lcc}
    \hline
        Subsystem & ITER Costs (M USD) & Scaled Costs (M USD)\\
        \hline
       22.5.1  & 29.3 & 13.9\\
       22.5.2  & 10.0 & 4.8\\
       22.5.3  & 32.2 & 15.3\\
       22.5.4  & 14.0 & 6.7\\
       22.5.5  & 32.6 & 15.6\\
       22.5.6  & 68.0 & 32.4\\
       22.5    & 186.0   & 88.7\\
       \hline
    \end{tabular}
    \caption{Breakdown of fuel handling and storage subsystems for ITER and the concept presented in this report.}
    \label{tab:fuel}
\end{table}
