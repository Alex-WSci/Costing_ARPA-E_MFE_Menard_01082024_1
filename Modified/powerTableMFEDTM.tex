%needs packages \usepackage{graphicx}   \usepackage{float} 
\newpage
\section{Power Accounting Table}

\begin{table}[ht!]								
\centering								
\begin{tabular}{|c|p{5cm}|c|c|c|}								
\hline								
\textbf{Account}	&	\textbf{Power Account Description}	&	\textbf{Parameter }	&	\textbf{Power}	&	\textbf{Units} \\
\hline								
1	&	Output power	&		&		&	\\
\hline
1.1	&	Fusion Power	&	$P_{{fusion}}$	&	2600	&	MW \\
1.2	&	Alpha Power	&	$P_{{\alpha}}$	&	520.6	&	MW \\
1.3	&	Neutron Power	&	$P_{{neutrons}}$	&	2079.4	&	MW \\
1.4	&	Neutron Energy Multiplier	&	$M_N = P_{{TH}}/P_{{fusion}}$	&	1.1	&	\\
1.5	&	Pumping power capture efficiency	&	$\eta_{{pump}}$	&	0.5	&	\\
1.6	&	Thermal Power	&	$P_{{TH}}$	&	2406.0	&	MW \\
1.7	&	Thermal conversion efficiency	&	$\eta_{{TH}}$	&	0.5	&	\\
1.8	& Total thermal electric power &	$P_{THe}$ &	1203.0	& MW \\
1.9	& Direct conversion efficiency &	$\eta_{DE}$ &	0.8 & \\	
1.10 &	Total direct electric power	& $P_{DEe}$	& 442.5 & MW \\ 
1.8	&	Total (Gross) Electric Power	&	$P_{{ET}}$	&	1645.5	&	MW \\
1.9	&	Lost Power	&	$P_{{Lost}}$	&	760.5	&	MW \\
\hline								
2	&	Recirculating power	&		&		&	\\
\hline
2.1	&	Power into coils 	&	$P_{{coils}} = P_{{Mirror}} + P_{{Solenoid}}$	&	2	&	MW \\
2.1.1	&	Power into Mirror coils	&	$P_{{Mirror}}$	&	1	&	MW \\
2.1.2	&	Power into Solenoid coils	&	$P_{{Solenoid}_e}$	&	1		&	MW \\
2.2	&	Primary Coolant Pumping Power Fraction	&	$f_{{pump}}$	&	0.1 &	\\
2.2.1	&	Primary Coolant Pumping Power	&	$P_{{pump}} = f_{{pump}} \cdot P_{{ET}}$	&	72.2	&	MW \\
2.3	&	Subsystem and Control Fraction	&	$f_{{sub}}$	&	0.0	&	\\
2.3.1	&	Subsystem and Control Power	&	$P_{{sub}} + P_{{control}} = f_{{sub}} \cdot P_{{ET}}$	&	36.1	&	MW \\
2.4	&	Auxiliary systems	&	$P_{{aux}} = P_{{t}} + P_{{h}}$	&	14.0	&	MW \\
2.4.1	&	Tritium Systems	&	$P_{{t}}$	&	10.0	&	MW \\
2.4.2	&	Housekeeping power	&	$P_{{h}}$	&	4.0	&	MW \\
2.5	&	Cooling systems	&	$P_{{cool}} = P_{{Mirror}_c} + P_{{Solenoid}_c}$	&	23.7	&	MW \\
2.5.1	&	Mirror coil cooling	&	$P_{{Mirror}_c}$	&	12.7	&	MW \\
2.5.2	&	Solenoid coil cooling	&	$P_{{Solenoid}_c}$	&	11	&	MW \\
2.5.3	&	Cryo vacuum pumping	&	$P_{{p}_c}$	&	0.5	&	MW \\
2.6.1	& Input power wall plug efficiency &	$\eta_{IN}$ & 0.5	&	MW \\
2.6.2	& Input power	& $P_{IN} = P_{heat} + P_{cd}$	&	50	&	MW \\
\hline								
3	&	Outputs	&		&		&	\\
\hline
3.1	&	Scientific Q	&	$Q = P_{{fusion}}/P_{{IN}}$	&	52.0	&	\\
3.2	&	Engineering Q	&	$Q_{{E}}$	&	6.6	&	\\
3.3	&	Recirculating power fraction	&	$\epsilon = 1/Q_{{E}}$	&	0.2	&	\\
3.4	&	Output Power (Net Electric Power)	&	$P_{{E}} = (1 - \epsilon) \cdot P_{{ET}}$	&	1397.3	&	MW \\
\hline								
\end{tabular}	
\caption{Power balance for a steady-state magnetic fusion energy system.}
\label{tab:powerbalance}
\end{table}

%With certain terms assumed as nominal values and a target recirculating power fraction taken to be $\epsilon$ = 0.20, a fusion power plant design space can be defined in terms of thermal conversion efficiency, $\eta_{TH}$ , and input efficiency, $\eta_{IN}$, in order to display isoquants of the necessary value of fusion 200.00, Q.  This design space is generalized; power terms do not appear explicitly, but once any specific power is selected, all other power terms are determined.  The highest values of $\eta_{IN}$ may be impractical and values of $\eta_{TH}$ approaching 0.60 require advanced Brayton-cycle technology.  The fusion power-plant design space is shown in Fig. \ref{fig:pbal}. With advanced thermal-conversion features \cite{Dabiri1989}, including a Brayton cycle, $\eta_{TH}$ can approach 0.60, or even 0.65 - 0.75 using a direct energy converter. A conventional Rankine thermal cycle should yield $\\eta_{TH}$ in the range 0.40 - 0.45.


%\begin{figure}[h!] 
%\centering
%\includegraphics[scale=0.6]{Figure3.eps}
%\caption{Power-Balance design space.}
%\label{fig:pbal}
%\end{figure}


%Fig \ref{fig:pbal} depicts the design space in terms of $\eta_{IN}$ versus $\eta_{TH}$ for the indicated fixed parameters with recirculating power fraction less that 20 percent. 
