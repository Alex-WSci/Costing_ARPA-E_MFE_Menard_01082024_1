\subsubsection*{Cost Category 22.01.04: Supplementary Heating Systems.}

Supplementary heating systems in fusion, such as neutral beam injection and radio frequency (RF) heating, are critical for achieving the necessary plasma conditions for fusion. Neutral beam injection involves injecting high-energy neutral atoms into the plasma, on the other hand, RF heating uses electromagnetic waves to transfer energy to the plasma. These systems are essential for heating the plasma to the extremely high temperatures required for fusion reactions to occur efficiently.\\

Supplementary heating systems contains costs associated with heating and current drive systems, including 

\begin{itemize}
    \item 22.01.04.01 Neutral Beam Heating.   Neutral Beam Injection (NBI): This method involves injecting high-energy neutral atoms (typically hydrogen or deuterium) into the plasma. These neutral atoms penetrate the magnetic fields and, upon ionization, transfer their energy to the plasma particles, thereby heating the plasma.
    \item 22.01.04.02 RF Heating.  Radio Frequency (RF) Heating: Uses electromagnetic waves to transfer energy directly to the plasma particles. Common methods include Ion Cyclotron Resonance Heating (ICRH), where ions absorb energy from the RF waves, and Electron Cyclotron Resonance Heating (ECRH), which targets electrons.
    \item 22.01.04.03 Laser heating.  Can include laser heating of Maglif targets, or precursor lasers for shaping of the target.
    \item 22.01.04.04 Other heating.  Can include schemes like novel electrostatic (biasing electrodes, pulsed, or oscillating electric fields) or time-varying magnetic (e.g. transit-time magnetic pumping).
\end{itemize}


The concept presented here will employ both NBI and ICRF supplementary heating systems. By comparing with prior studies (see table \ref{tab:supp_heat}), a top-down estimation can be found for each. For a 50 MW NBI system, the cost is 353 M USD, and for a ICRFpower MW ICRF, the cost is 0 M USD, totalling 353 MUSD for the system.

\begin{table}[htbp]
    \centering
   
    \begin{tabular}{lcccrr}
        \hline
        System & Type & Power (MW) & \$/W (2009) & \$/W (2023) \\
        \hline
        ARIES-AT & ICRF/LH & 37.4 & 1.67 & 2.39 \\
        ARIES-I & ICRF/LH & 96.7 & 1.87 & 2.67 \\
        ARIES-I' & ICRF/LH & 202 & 1.96 & 2.80 \\
        ARIES-RS & ICRF/LH/HFFW & 80.8 & 3.09 & 4.42 \\
        ARIES-IV & ICRF/LH & 68.0 & 4.35 & 6.22 \\
        ARIES-II & ICRF/LH & 66.1 & 4.47 & 6.39 \\
        ARIES-III' & NBI & 163 & 4.93 & 7.05 \\
        ARIES-III & NBI & 172 & 4.95 & 7.08 \\
        ITER & ICRF & & 5.5 & 7.87 \\
        Average &  & 110 & 3.64 & 5.21 \\
        Average (ICRF) &  & 91.9 & 2.90 & 4.15 \\
        Average (NBI) &  & 167 & 4.94 & 7.06 \\
        \hline
    \end{tabular}
     \caption{Power and Cost Data (Rounded to 3 Significant Figures)}
    \label{tab:supp_heat}
\end{table}

