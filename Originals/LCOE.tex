\newpage 

\section{Cost of Electricity} 

 Contributors to LCOE projections are given by: 
 \begin{equation} 
 LCOE [\$ MWh] = \frac{(C_{AC} + (C_{OM} + C_{F})(1+y)^Y)}{(8760.P_E.p_a)} 
 \label{eq:coe}
\end{equation} 
where C$_{AC}$ [USD/year] is the annual capital cost charge (entailing the total capital cost of the plant (TCC (USD) multiplied by the Fixed Charge Rate (FCR (/year)), C$_{OM}$ [USD/year] is the annual operations and maintenance cost, C$_{SCR}$ [USD/year] is the annual scheduled component replacement costs, C$_{F}$ [USD/year] is the annual fuel costs, $y$ is the annual fractional increase in fuel costs over the expected lifetime of the plant $Y$ [years], P$_{E}$ [MWe] is the net electric power of the plant, p$_{a}$ is the plant availability (typically 0.6-0.9).  A small charge used to be imposed to build up a fund to cover end-of-life Decontamination and Decommissioning ($DD$), f$_{DD}$ (USD/kWh), but the DD costs are now included in the direct capitalized costs (Cost Category 58), so are omitted here.

\subsection{Levelized Cost of Electricity} 

\begin{table}[h!] 
\begin{tabular}{l c c } 
Net electric power & MW & PNET \\
Plant availability & \% & PAVAIL \\
Inflation & \% & yinflation \\
Lifetime of plant & Years & lifeY \\
Capital cost, $C_{AC}$ &     M\$/annum    &    C900000      \\ 
Operations and Maintenance Costs, $C_{OM}$ & M\$/annum  &      C700000 \\ 
Fuel Costs, $C_{F}$ & M\$/annum  &       C800000 \\ 
    \end{tabular} 
    \caption{Variables in the LCOE calculation.}
    \label{tab:lcoe} 
\end{table} 

Following equation \ref{eq:coe}, the LCOE is C1000000 \$/MWh or C2000000 c/kWh. 

%\subsection{Levelized Cost of Electricity Nth of a Kind (NOAK)}  

%It is possible to distinguish between the First of a Kind (FOAK) power plant, which might function as a Demo, and more mature 10th of a Kind (TOAK), the latter incorporating learning credits in the COE estimate, consistent with U.S. fusion-reactor design-community practice. The ARIES study, like STARFIRE \cite{BAK80} and most other U.S. fusion-reactor designs reported in the last decade, assumes FPC unit costs consistent with these learning-curve \cite{Hir64,Arg90} credits, rather than first-of-a-kind unit costs (including R\&D) appropriate for ITER \cite{Ite89} or some other reported designs \cite{Coo89b}. An 80 \% learning curve ({\it {\it i.e.},} 0.80 progress ratio, $p$), as used for ARIES, represents the expectation that each doubling of production represents a \hbox{$p\simeq$ 0.8} reduction in unit costs. A tenth-of-a-kind reactor represents, nominally, $\simeq$3.3 doubling of production or $p^{(ln 10/ln 2)}\simeq$ 50 \% cost reduction relative to first-of-a-kind FPC costs.  Of course, the actual production experience varies \cite{Arg90}.  Learning credits are not applied to BOP items, consistent with mature industrial production in which the learning credits have already been wrung out.  For similar reasons, a 94 \% learning curve is recommended for advanced fission cost estimates \cite{Del93}.   