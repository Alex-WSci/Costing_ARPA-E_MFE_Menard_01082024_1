\section{Cost Category 50: Capitalized Supplementary Costs (CSC)}


\subsection*{Cost Category 51 – Shipping and Transportation Costs}
This Cost Category includes shipping and transportation costs for major equipment or bulk shipments with freight forwarding. Budget is \$ 8 M.

\subsection*{Cost Category 52 – Spare Parts}
This Cost Category includes spare parts furnished by system suppliers for the first year of commercial operation. It excludes spare parts required for plant commissioning, startup, or the demonstration run.  Budget is \$ 12 M, which is calculated on the basis of 10 \% of the direct cost components in the BOP (note that Scheduled Replacement Costs for the heat island are calculated in Cost Category 75.

\subsection*{Cost Category 53 – Taxes}
This Cost Category includes taxes associated with the permanent plant, such as property tax, to be capitalized with the plant. Budget is \$ 100 M.

\subsection*{Cost Category 54 – Insurance}
This Cost Category includes insurance costs associated with the permanent plant to be capitalized with the plant.  Budget is \$ 1 M.

\subsection*{Cost Category 55 – Initial Fuel Load}
This Cost Category covers fuel purchased by the utility before commissioning, which is assumed to be part of the TCIC. For DT systems this will include the initial tritium startup costs.  Due to its scarcity, tritium costs \$109,570 to \$169,570 per gram in 2016 USD. Therefore, the onetime start-up tritium cost ranges from \$22 M to \$34 M (2016 USD) for a standard 150 MWe FPP. \\

We allocate a budget of \$ 14 M for the initial fuel load.  

\subsection*{Cost Category 58 – Decommissioning Costs}
This Cost Category includes the cost to decommission, decontaminate, and dismantle the plant at the end of commercial operation, if it is capitalized with the plant.  This cost used to be broken out as a separate line item in the cost of electricity calculation.\\

The D\&D allowance are assumed flat and constant for the systems under consideration here, and we base the amount on the following:

\begin{verbatim} 
Decontamination and Decommissioning Allowance 
TFTR (51 MWth) $36.7M (Actual 2002$) 
Generic Magnetic Fusion Reactor 0.5 M/kWh (1983$) 
Light Water Reactor $307M - $819M (Actual 2013$) 
Generic Magnetic Fusion Reactor Revisited, J Sheffield/S Milora,  
ANS Fusion Science and Technology Vol.70, 14-35, July 2016. 
Decommissioning of the Tokamak Fusion Test Reactor,
Perry/Chrzanowski/Gentile/Parsells/Rule/Strykowsky/Viola, 
Princeton Plasma Physics Laboratory, PPPL-3895, October 2003.
Costs of Decommissioning Nuclear Power Plants, NEA No. 7201. OECD 2016.
\end{verbatim} 

Prior to 2007, these costs were included in the LCOE as a budget of 0.5mills/kWh.  Here, we provide a budget estimate for the decommissioning costs of \$ 200 M, which is consistent with current D\&D costs for SMRs. 

\subsection*{Cost Category 59 – Contingency on Supplementary Costs}
This Cost Category includes an assessment of additional cost necessary to achieve a desired confidence level for the capitalized supplementary costs not to be exceeded. The contingency for the initial core load should not be applied to this item, because the contingency is already embedded in the fuel cycle costs from the fuel cycle model.\\

Budget is \$ 0 M, based on 10 \% of the sum of the CSC.
