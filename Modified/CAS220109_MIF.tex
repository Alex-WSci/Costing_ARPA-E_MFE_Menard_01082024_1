\subsubsection*{Cost Category 22.1.9: Direct Energy Convertor}

The direct energy converter consists broadly of an expander and collector, comprising a grid of ribbons arranged in a 'venetian blind' configuration \cite{post1970mirror}. Various expander geometries have been proposed \cite{post1970mirror}. A conical configuration was selected, due to the lower cost below 800 keV, as well as the ability to maintain the individual modules \cite{barr1974preliminary}. This direct energy conversion has an efficiency of $\sim$ 60 - 70\% \cite{moir1973venetian}.\\

The collector can accrue damage from sputtering caused by the ion fluence. For tungsten ribbons, this erosion rate is estimated to be 0.02mm/year, resulting in a low likelihood of having to replace grid elements.\\

Using additively manufactured ribbons with a simple water coolant channel, a heat flux of 2MW/m$^3$ is acceptable, reducing the overall volume and thus the cost of the subsystem by a factor of 60\% compared with a 1MW/m$^3$ flux radiatively-cooled system.\\

 Thus, with the converter structure, the total cost is 209 MUSD.


\begin{table}[h]
    \centering
    \resizebox{0.6\linewidth}{!}{%
    \begin{tabular}{|l|r|}
    \hline
    \textbf{Item} & \textbf{Cost (M USD)} \\ \hline
    Expander tank & 5.7 \\ \hline
    Expander Coil and Neutron Trap Coil & 11.7\\ \hline
    Convertor gate valve & 0.0 \\ \hline
    Neutron Trap Shielding & 0.4 \\ \hline
    Vacuum system & 5.7 \\ \hline
    Grid system & 9.5 \\ \hline
    Heat collection system & 2.1 \\ \hline
    Electrical & 4.6 \\ \hline
    Cost per unit & 39.6 \\ \hline
    Total unit cost & 158.0 \\ \hline
    Engineering (15\%) & 23.7 \\ \hline
    Contingency (15\%) & 27.2 \\ \hline
    Total facility cost & 209.0\\ \hline
    \end{tabular}}
    \caption{Cost Table}
    \label{tab:cost-table}
\end{table}
