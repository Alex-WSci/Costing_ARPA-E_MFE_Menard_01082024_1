
\subsubsection{Cost Category 22.04: Radioactive Waste Treatment} 
In the realm of managing radioactive materials in fusion energy systems, several key processes are pivotal. Firstly, the purification of the liquid breeder or heat transfer media is essential to maintain the concentration of radioactive products, specifically primaryC, within safe limits.  Radioactive materials necessitate careful classification for either recycling, clearance, or disposal in appropriate off-site repositories.  According to El-Guebaly et al. in “Goals, Challenges, and Successes of Managing Fusion Active Materials,” \cite{Elguebaly} radioactive materials must be prepared for either recycling, clearance, or proper disposal in designated off-site repositories.  The processing equipment is likely located in the Hot Cell building. This equipment encompasses remote handling tools, storage tanks, pumps, piping, valves, heat exchangers, heaters, condensers, gas strippers, compressors, chemical reactors, evaporators, ion exchange subsystems, filters, traps, and separators.  The on-site system is designed for basic processing and management, not for extensive processing. It does not serve as the primary system for online separation of Deuterium and Tritium fuel isotopes, which is handled by the Fuel Handling and Storage system (Account 22.5). Nonetheless, cost-effectively recovered tritium, such as from detritiation processes, is reintegrated into the Fuel Handling and Storage system for further refinement.\\

Consists of: 

\begin{itemize}
\item Cost Category 22.04.01 liquid waste processing and equipment.  This involves purifying the liquid breeder or heat transfer medium,  maintaining the concentration of radioactive products, in this case primaryC, below specific safety limits.
\item Cost Category 22.04.02 gaseous wastes and off-gas processing system. It involves capturing, filtering, and treating off-gases to remove contaminants and reduce radioactive emissions. This process is vital for minimizing environmental impact and adhering to stringent safety standards.
\item Cost Category 22.04.03 solid waste processing equipment. It involves the segregation, treatment, and preparation of solid wastes for safe storage or disposal. This process is essential for ensuring long-term environmental safety and compliance with regulatory requirements.
\end{itemize}




It's important to note that fusion reactors produce less long-lived radioactive waste compared to fission reactors. The IAEA notes that fusion reactor waste is primarily composed of activated structural materials, and the volume of high-level waste is significantly less \cite{girard2008summary}. This makes waste management in fusion reactors potentially more manageable and less hazardous over the long term. The development of low-activation structural materials remains a key enabling technology for fusion implementation and public acceptance.\\

Costs are scaled relative to thermal power, per ARIES-ST and escalted relative to 1992 \$ \cite{DEL90b}. The resultant total cost is 2.1 M USD. 